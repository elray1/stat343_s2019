\documentclass[]{article}
\usepackage{lmodern}
\usepackage{amssymb,amsmath}
\usepackage{ifxetex,ifluatex}
\usepackage{fixltx2e} % provides \textsubscript
\ifnum 0\ifxetex 1\fi\ifluatex 1\fi=0 % if pdftex
  \usepackage[T1]{fontenc}
  \usepackage[utf8]{inputenc}
\else % if luatex or xelatex
  \ifxetex
    \usepackage{mathspec}
  \else
    \usepackage{fontspec}
  \fi
  \defaultfontfeatures{Ligatures=TeX,Scale=MatchLowercase}
\fi
% use upquote if available, for straight quotes in verbatim environments
\IfFileExists{upquote.sty}{\usepackage{upquote}}{}
% use microtype if available
\IfFileExists{microtype.sty}{%
\usepackage{microtype}
\UseMicrotypeSet[protrusion]{basicmath} % disable protrusion for tt fonts
}{}
\usepackage[margin=1.5cm]{geometry}
\usepackage{hyperref}
\hypersetup{unicode=true,
            pdftitle={Large-sample approximate confidence interval; example 2, Rayleigh distribution},
            pdfborder={0 0 0},
            breaklinks=true}
\urlstyle{same}  % don't use monospace font for urls
\usepackage{graphicx,grffile}
\makeatletter
\def\maxwidth{\ifdim\Gin@nat@width>\linewidth\linewidth\else\Gin@nat@width\fi}
\def\maxheight{\ifdim\Gin@nat@height>\textheight\textheight\else\Gin@nat@height\fi}
\makeatother
% Scale images if necessary, so that they will not overflow the page
% margins by default, and it is still possible to overwrite the defaults
% using explicit options in \includegraphics[width, height, ...]{}
\setkeys{Gin}{width=\maxwidth,height=\maxheight,keepaspectratio}
\IfFileExists{parskip.sty}{%
\usepackage{parskip}
}{% else
\setlength{\parindent}{0pt}
\setlength{\parskip}{6pt plus 2pt minus 1pt}
}
\setlength{\emergencystretch}{3em}  % prevent overfull lines
\providecommand{\tightlist}{%
  \setlength{\itemsep}{0pt}\setlength{\parskip}{0pt}}
\setcounter{secnumdepth}{0}
% Redefines (sub)paragraphs to behave more like sections
\ifx\paragraph\undefined\else
\let\oldparagraph\paragraph
\renewcommand{\paragraph}[1]{\oldparagraph{#1}\mbox{}}
\fi
\ifx\subparagraph\undefined\else
\let\oldsubparagraph\subparagraph
\renewcommand{\subparagraph}[1]{\oldsubparagraph{#1}\mbox{}}
\fi

%%% Use protect on footnotes to avoid problems with footnotes in titles
\let\rmarkdownfootnote\footnote%
\def\footnote{\protect\rmarkdownfootnote}

%%% Change title format to be more compact
\usepackage{titling}

% Create subtitle command for use in maketitle
\newcommand{\subtitle}[1]{
  \posttitle{
    \begin{center}\large#1\end{center}
    }
}

\setlength{\droptitle}{-2em}

  \title{Large-sample approximate confidence interval; example 2, Rayleigh
distribution}
    \pretitle{\vspace{\droptitle}\centering\huge}
  \posttitle{\par}
    \author{}
    \preauthor{}\postauthor{}
    \date{}
    \predate{}\postdate{}
  
\usepackage{booktabs}

\begin{document}
\maketitle

\def\simiid{\stackrel{{\mbox{\text{\tiny i.i.d.}}}}{\sim}}

\subsection{Spatial Organization of Chromosome (Rice Problem
8.45)}\label{spatial-organization-of-chromosome-rice-problem-8.45}

In problem 2 of problem set 3, we analyzed data from experiments that
were conducted to learn about the spatial organization of chromosomes.
We had \(n\) measurements of the distances between pairs of small DNA
sequences, which we modeled as \(X_i \simiid \text{Rayleigh}(\theta)\).

In that problem set, we found the maximum likelihood estimator and
estimate of \(\theta\); now, we will find an approximate confidence
interval for \(\theta\).

Here are some results we had in problem set 3:

If \(X \sim \text{Rayleigh}(\theta)\) (with parameter \(\theta > 0\)),
then the probability density function is given by

\[f(x | \theta) = \frac{x}{\theta} \exp \left( \frac{-x^2}{2 \theta} \right)\]

for positive values of \(x\) (and the probability density function is 0
for non-positive values of \(x\)).

We have the following results about the moments of a
Rayleigh-distributed random variable:

\begin{align*}
E(X) &= \left( \frac{\theta \pi}{2} \right)^{1/2} \\
E(X^2) &= 2 \theta \\
Var(X) &= 2 \theta - \frac{\theta \pi}{2}
\end{align*}

The likelihood is given by

\begin{align*}
\mathcal{L}(\theta | x_1, \ldots, x_n) &= f(x_1, \ldots, x_n | \theta) \\
 &= \prod_{i=1}^n f(x_i | \theta) \\
 &= \prod_{i=1}^n \frac{x_i}{\theta} \exp \left( \frac{-x_i^2}{2 \theta} \right)
\end{align*}

The log-likelihood is therefore

\begin{align*}
\ell(\theta | x_1, \ldots, x_n) &= \log\left\{ \prod_{i=1}^n \frac{x_i}{\theta} \exp \left( \frac{-x_i^2}{2 \theta} \right) \right\} \\
 &= \sum_{i=1}^n \left\{ \log(x_i) - \log(\theta) - \frac{x_i^2}{2 \theta} \right\} \\
 &= \sum_{i=1}^n \log(x_i) - n \log(\theta) - \frac{1}{2 \theta} \sum_{i=1}^n x_i^2
\end{align*}

\newpage

The first derivative of the log-likelihood is

\begin{align*}
\frac{d}{d \theta} \ell(\theta | x_1, \ldots, x_n) &= \frac{d}{d \theta} \left\{ \sum_{i=1}^n \log(x_i) - n \log(\theta) - \frac{1}{2\theta} \sum_{i=1}^n x_i^2 \right\} \\
&= \frac{-n}{\theta} + \frac{1}{2\theta^2} \sum_{i=1}^n x_i^2
\end{align*}

Setting this equal to 0 and solving for theta, we obtain a maximum
likelihood estimator of
\(\hat{\theta}_{MLE} = \frac{1}{2n} \sum_{i=1}^n X_{i}^2\).

The second derivative of the log-likelihood is:

\begin{align*}
\frac{d^2}{d \theta^2} \ell(\theta | x_1, \ldots, x_n) &= \frac{d}{d \theta} \left\{ \frac{-n}{\theta} + \frac{1}{2\theta^2} \sum_{i=1}^n x_i^2 \right\} \\
&= \frac{-n}{\theta^2}(-1) + \frac{1}{2\theta^3} (-2) \sum_{i=1}^n x_i^2 \\
&= \frac{n}{\theta^2} - \frac{1}{\theta^3} \sum_{i=1}^n x_i^2
\end{align*}

\paragraph{1. Find an expression for the observed Fisher information
(not much work to do
here\ldots{}).}\label{find-an-expression-for-the-observed-fisher-information-not-much-work-to-do-here.}

\vspace{3cm}

\paragraph{2. Find an expression for the Fisher
information.}\label{find-an-expression-for-the-fisher-information.}


\end{document}
