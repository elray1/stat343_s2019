\documentclass[]{article}
\usepackage{lmodern}
\usepackage{amssymb,amsmath}
\usepackage{ifxetex,ifluatex}
\usepackage{fixltx2e} % provides \textsubscript
\ifnum 0\ifxetex 1\fi\ifluatex 1\fi=0 % if pdftex
  \usepackage[T1]{fontenc}
  \usepackage[utf8]{inputenc}
\else % if luatex or xelatex
  \ifxetex
    \usepackage{mathspec}
  \else
    \usepackage{fontspec}
  \fi
  \defaultfontfeatures{Ligatures=TeX,Scale=MatchLowercase}
\fi
% use upquote if available, for straight quotes in verbatim environments
\IfFileExists{upquote.sty}{\usepackage{upquote}}{}
% use microtype if available
\IfFileExists{microtype.sty}{%
\usepackage{microtype}
\UseMicrotypeSet[protrusion]{basicmath} % disable protrusion for tt fonts
}{}
\usepackage[margin=1.5cm]{geometry}
\usepackage{hyperref}
\hypersetup{unicode=true,
            pdftitle={Large-sample approximate sampling distribution of the MLE: Pareto Distribution},
            pdfborder={0 0 0},
            breaklinks=true}
\urlstyle{same}  % don't use monospace font for urls
\usepackage{graphicx,grffile}
\makeatletter
\def\maxwidth{\ifdim\Gin@nat@width>\linewidth\linewidth\else\Gin@nat@width\fi}
\def\maxheight{\ifdim\Gin@nat@height>\textheight\textheight\else\Gin@nat@height\fi}
\makeatother
% Scale images if necessary, so that they will not overflow the page
% margins by default, and it is still possible to overwrite the defaults
% using explicit options in \includegraphics[width, height, ...]{}
\setkeys{Gin}{width=\maxwidth,height=\maxheight,keepaspectratio}
\IfFileExists{parskip.sty}{%
\usepackage{parskip}
}{% else
\setlength{\parindent}{0pt}
\setlength{\parskip}{6pt plus 2pt minus 1pt}
}
\setlength{\emergencystretch}{3em}  % prevent overfull lines
\providecommand{\tightlist}{%
  \setlength{\itemsep}{0pt}\setlength{\parskip}{0pt}}
\setcounter{secnumdepth}{0}
% Redefines (sub)paragraphs to behave more like sections
\ifx\paragraph\undefined\else
\let\oldparagraph\paragraph
\renewcommand{\paragraph}[1]{\oldparagraph{#1}\mbox{}}
\fi
\ifx\subparagraph\undefined\else
\let\oldsubparagraph\subparagraph
\renewcommand{\subparagraph}[1]{\oldsubparagraph{#1}\mbox{}}
\fi

%%% Use protect on footnotes to avoid problems with footnotes in titles
\let\rmarkdownfootnote\footnote%
\def\footnote{\protect\rmarkdownfootnote}

%%% Change title format to be more compact
\usepackage{titling}

% Create subtitle command for use in maketitle
\newcommand{\subtitle}[1]{
  \posttitle{
    \begin{center}\large#1\end{center}
    }
}

\setlength{\droptitle}{-2em}

  \title{Large-sample approximate sampling distribution of the MLE: Pareto
Distribution}
    \pretitle{\vspace{\droptitle}\centering\huge}
  \posttitle{\par}
    \author{}
    \preauthor{}\postauthor{}
    \date{}
    \predate{}\postdate{}
  
\usepackage{booktabs}

\begin{document}
\maketitle

\def\simiid{\stackrel{{\mbox{\text{\tiny i.i.d.}}}}{\sim}}

\section{Stat 343 Lab: Sampling Distribution of the
MLE}\label{stat-343-lab-sampling-distribution-of-the-mle}

The Pareto distribution is often used to model the distribution of
variables with long right tail; for example, one recent article used the
Pareto distribution (and several other candidate models) to model the
distribution of carbon dioxide emissions by countries around the world
(Akhundjanov, S. B., Devadoss, S., \& Luckstead, J. (2017). Size
distribution of national CO2 emissions. Energy Economics, 66182-193).

The Pareto distribution has two parameters, \(x_0 > 0\) and
\(\alpha > 0\). If \(X \sim \text{Pareto}(x_0, \alpha)\), then \(X\) is
a continuous random variable. Its probability density function (pdf) is:

\(f(x | x_0, \alpha) = \frac{\alpha x_0^\alpha}{x^{\alpha + 1}}\mathbb{I}_{[x_0, \infty)}(x) = \begin{cases} \frac{\alpha x_0^\alpha}{x^{\alpha + 1}} \text{ if $x \geq x_0$} \\ 0 \text{ otherwise} \end{cases}\)

The mean of \(X\) is \(E(X) = \frac{\alpha x_0}{\alpha - 1}\) if
\(\alpha > 1\), and infinity otherwise.

Let's fix \(x_0 = 1\) and think about how we might estimate the
parameter \(\alpha\).

\subsection{Strategy 1: Method of
Moments}\label{strategy-1-method-of-moments}

One approach to estimation that we have skipped over so far is the
method of moments. The method of moments works by setting the first few
moments of a distribution equal to the corresponding moments of a data
set, and solving for any unknown parameters. In this case, that means
setting the sample mean equal to the mean of the Pareto distribution and
solving for \(\alpha\). Go ahead and do that to find an estimator of
\(\alpha\), \(\hat{\alpha}^{MOM}\). It will be a function of the sample
mean \(\bar{X}\).

\newpage

\subsection{Strategy 2: Maximum
Likelihood}\label{strategy-2-maximum-likelihood}

I'll let you skip a few steps (but on the other hand, here is a chance
to get some practice if the mechanics of these steps are not yet
comfortable).

The log-likelihood for a Pareto distribution is given by

\(\ell(\alpha | x_1, \ldots, x_n, x_0) = n \log(\alpha) + n \alpha \log(x_0) - (\alpha + 1) \sum_{i = 1}^n \log(x_i)\).

The first derivative of the log-likelihood is

\(\frac{d}{d\alpha} \ell(\alpha | x_1, \ldots, x_n, x_0) = \frac{n}{\alpha} + n \log(x_0) - \sum_{i = 1}^n \log(x_i)\).

The second derivative of the log-likelihood is

\(\frac{d^2}{d\alpha^2} \ell(\alpha | x_1, \ldots, x_n, x_0) = \frac{-n}{\alpha^2}\).

Note that when \(x_0 = 1\), \(\log(x_0) = 0\).

\textbf{Find the maximum likelihood estimator}

\vspace{6cm}

\subsection{\texorpdfstring{Asymptotic Distribution of the MLE, assuming
\(\alpha^{true} = 1.1\)}{Asymptotic Distribution of the MLE, assuming \textbackslash{}alpha\^{}\{true\} = 1.1}}\label{asymptotic-distribution-of-the-mle-assuming-alphatrue-1.1}

Find a normal approximation to the sampling distribution of
\(\hat{\alpha}^{MLE}\). This will depend on the sample size \(n\). Is
there any difference between the observed Fisher information and the
Fisher information in this case?


\end{document}
