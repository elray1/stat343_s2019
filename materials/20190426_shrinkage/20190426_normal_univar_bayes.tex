\documentclass[]{article}
\usepackage{lmodern}
\usepackage{amssymb,amsmath}
\usepackage{ifxetex,ifluatex}
\usepackage{fixltx2e} % provides \textsubscript
\ifnum 0\ifxetex 1\fi\ifluatex 1\fi=0 % if pdftex
  \usepackage[T1]{fontenc}
  \usepackage[utf8]{inputenc}
\else % if luatex or xelatex
  \ifxetex
    \usepackage{mathspec}
  \else
    \usepackage{fontspec}
  \fi
  \defaultfontfeatures{Ligatures=TeX,Scale=MatchLowercase}
\fi
% use upquote if available, for straight quotes in verbatim environments
\IfFileExists{upquote.sty}{\usepackage{upquote}}{}
% use microtype if available
\IfFileExists{microtype.sty}{%
\usepackage{microtype}
\UseMicrotypeSet[protrusion]{basicmath} % disable protrusion for tt fonts
}{}
\usepackage[margin=1in]{geometry}
\usepackage{hyperref}
\hypersetup{unicode=true,
            pdftitle={Bias, Variance, and MSE of Maximum Likelihood and Bayesian Estimators of the Mean of a Univariate Normal Distribution},
            pdfborder={0 0 0},
            breaklinks=true}
\urlstyle{same}  % don't use monospace font for urls
\usepackage{graphicx,grffile}
\makeatletter
\def\maxwidth{\ifdim\Gin@nat@width>\linewidth\linewidth\else\Gin@nat@width\fi}
\def\maxheight{\ifdim\Gin@nat@height>\textheight\textheight\else\Gin@nat@height\fi}
\makeatother
% Scale images if necessary, so that they will not overflow the page
% margins by default, and it is still possible to overwrite the defaults
% using explicit options in \includegraphics[width, height, ...]{}
\setkeys{Gin}{width=\maxwidth,height=\maxheight,keepaspectratio}
\IfFileExists{parskip.sty}{%
\usepackage{parskip}
}{% else
\setlength{\parindent}{0pt}
\setlength{\parskip}{6pt plus 2pt minus 1pt}
}
\setlength{\emergencystretch}{3em}  % prevent overfull lines
\providecommand{\tightlist}{%
  \setlength{\itemsep}{0pt}\setlength{\parskip}{0pt}}
\setcounter{secnumdepth}{0}
% Redefines (sub)paragraphs to behave more like sections
\ifx\paragraph\undefined\else
\let\oldparagraph\paragraph
\renewcommand{\paragraph}[1]{\oldparagraph{#1}\mbox{}}
\fi
\ifx\subparagraph\undefined\else
\let\oldsubparagraph\subparagraph
\renewcommand{\subparagraph}[1]{\oldsubparagraph{#1}\mbox{}}
\fi

%%% Use protect on footnotes to avoid problems with footnotes in titles
\let\rmarkdownfootnote\footnote%
\def\footnote{\protect\rmarkdownfootnote}

%%% Change title format to be more compact
\usepackage{titling}

% Create subtitle command for use in maketitle
\newcommand{\subtitle}[1]{
  \posttitle{
    \begin{center}\large#1\end{center}
    }
}

\setlength{\droptitle}{-2em}

  \title{Bias, Variance, and MSE of Maximum Likelihood and Bayesian Estimators of
the Mean of a Univariate Normal Distribution}
    \pretitle{\vspace{\droptitle}\centering\huge}
  \posttitle{\par}
    \author{}
    \preauthor{}\postauthor{}
    \date{}
    \predate{}\postdate{}
  

\begin{document}
\maketitle

\def\simiid{\stackrel{{\mbox{\text{\tiny i.i.d.}}}}{\sim}}

\subsection{Case 1: Bayesian Estimator has larger bias away from the
prior mean, lower variance everywhere, lower MSE near the prior
mean}\label{case-1-bayesian-estimator-has-larger-bias-away-from-the-prior-mean-lower-variance-everywhere-lower-mse-near-the-prior-mean}

\begin{itemize}
\tightlist
\item
  \(n\) = 10
\item
  \(\gamma_{prior} = 0\)
\item
  \(\xi_{prior} = 1\)
\item
  \(\xi = 1\)
\end{itemize}

With these settings, the MLE is \(\bar{X}\) and the Bayesian estimator
is \(\frac{10}{11} \bar{X}\)

\includegraphics{20190426_normal_univar_bayes_files/figure-latex/unnamed-chunk-1-1.pdf}

\newpage

\subsection{Case 2: Location where Bayesian Estimator has lower MSE
depends on the prior
mean.}\label{case-2-location-where-bayesian-estimator-has-lower-mse-depends-on-the-prior-mean.}

\begin{itemize}
\tightlist
\item
  \(n = 10\)
\item
  \(\gamma_{prior} = 2.5\)
\item
  \(\xi_{prior} = 1\)
\item
  \(\xi = 1\)
\end{itemize}

With these settings, the MLE is \(\bar{X}\) and the Bayesian estimator
is \(\frac{10}{11} \bar{X} + \frac{1}{11}2.5\)

\includegraphics{20190426_normal_univar_bayes_files/figure-latex/unnamed-chunk-2-1.pdf}

\newpage

\subsection{\texorpdfstring{Case 3: Range of values where Bayesian
Estimator has lower MSE depends on the relative sizes of \(\xi\) and
\(\xi_{prior}\).}{Case 3: Range of values where Bayesian Estimator has lower MSE depends on the relative sizes of \textbackslash{}xi and \textbackslash{}xi\_\{prior\}.}}\label{case-3-range-of-values-where-bayesian-estimator-has-lower-mse-depends-on-the-relative-sizes-of-xi-and-xi_prior.}

\begin{itemize}
\tightlist
\item
  \(n = 10\)
\item
  \(\gamma_{prior} = 0\)
\item
  \(\xi_{prior} = 0.25\)
\item
  \(\xi = 1\)
\end{itemize}

With these settings, the MLE is \(\bar{X}\) and the Bayesian estimator
is \(\frac{10}{10.25} \bar{X}\)

\includegraphics{20190426_normal_univar_bayes_files/figure-latex/unnamed-chunk-3-1.pdf}


\end{document}
