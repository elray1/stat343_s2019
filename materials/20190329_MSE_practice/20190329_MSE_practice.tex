\documentclass[]{article}
\usepackage{lmodern}
\usepackage{amssymb,amsmath}
\usepackage{ifxetex,ifluatex}
\usepackage{fixltx2e} % provides \textsubscript
\ifnum 0\ifxetex 1\fi\ifluatex 1\fi=0 % if pdftex
  \usepackage[T1]{fontenc}
  \usepackage[utf8]{inputenc}
\else % if luatex or xelatex
  \ifxetex
    \usepackage{mathspec}
  \else
    \usepackage{fontspec}
  \fi
  \defaultfontfeatures{Ligatures=TeX,Scale=MatchLowercase}
\fi
% use upquote if available, for straight quotes in verbatim environments
\IfFileExists{upquote.sty}{\usepackage{upquote}}{}
% use microtype if available
\IfFileExists{microtype.sty}{%
\usepackage{microtype}
\UseMicrotypeSet[protrusion]{basicmath} % disable protrusion for tt fonts
}{}
\usepackage[margin=1.5cm]{geometry}
\usepackage{hyperref}
\hypersetup{unicode=true,
            pdftitle={Bayesian Inference, Shrinkage Estimators, and MSE},
            pdfborder={0 0 0},
            breaklinks=true}
\urlstyle{same}  % don't use monospace font for urls
\usepackage{graphicx,grffile}
\makeatletter
\def\maxwidth{\ifdim\Gin@nat@width>\linewidth\linewidth\else\Gin@nat@width\fi}
\def\maxheight{\ifdim\Gin@nat@height>\textheight\textheight\else\Gin@nat@height\fi}
\makeatother
% Scale images if necessary, so that they will not overflow the page
% margins by default, and it is still possible to overwrite the defaults
% using explicit options in \includegraphics[width, height, ...]{}
\setkeys{Gin}{width=\maxwidth,height=\maxheight,keepaspectratio}
\IfFileExists{parskip.sty}{%
\usepackage{parskip}
}{% else
\setlength{\parindent}{0pt}
\setlength{\parskip}{6pt plus 2pt minus 1pt}
}
\setlength{\emergencystretch}{3em}  % prevent overfull lines
\providecommand{\tightlist}{%
  \setlength{\itemsep}{0pt}\setlength{\parskip}{0pt}}
\setcounter{secnumdepth}{0}
% Redefines (sub)paragraphs to behave more like sections
\ifx\paragraph\undefined\else
\let\oldparagraph\paragraph
\renewcommand{\paragraph}[1]{\oldparagraph{#1}\mbox{}}
\fi
\ifx\subparagraph\undefined\else
\let\oldsubparagraph\subparagraph
\renewcommand{\subparagraph}[1]{\oldsubparagraph{#1}\mbox{}}
\fi

%%% Use protect on footnotes to avoid problems with footnotes in titles
\let\rmarkdownfootnote\footnote%
\def\footnote{\protect\rmarkdownfootnote}

%%% Change title format to be more compact
\usepackage{titling}

% Create subtitle command for use in maketitle
\newcommand{\subtitle}[1]{
  \posttitle{
    \begin{center}\large#1\end{center}
    }
}

\setlength{\droptitle}{-2em}

  \title{Bayesian Inference, Shrinkage Estimators, and MSE}
    \pretitle{\vspace{\droptitle}\centering\huge}
  \posttitle{\par}
    \author{}
    \preauthor{}\postauthor{}
    \date{}
    \predate{}\postdate{}
  
\usepackage{booktabs}

\begin{document}
\maketitle

\def\simiid{\stackrel{{\mbox{\text{\tiny i.i.d.}}}}{\sim}}

\section{M\&M's Example: Bias, Variance, Efficiency, and MSE of maximum
likelihood estimator and Bayesian posterior
mean}\label{mms-example-bias-variance-efficiency-and-mse-of-maximum-likelihood-estimator-and-bayesian-posterior-mean}

\subsection{Set Up}\label{set-up}

\paragraph{Problem Reminder}\label{problem-reminder}

In lab 7, we estimated the proportion of M\&M's that are blue,
\(\theta\), based on a sample of \(n\) M\&M's. We defined the random
variable \(X\), which was the count of how many M\&M's were blue in that
sample. Our model was \(X \simiid \text{Binomial}(n, \theta)\).

We have developed two approaches to inference for \(\theta\):

\begin{enumerate}
\def\labelenumi{\arabic{enumi}.}
\item
  The maximum likelihood estimator \(\hat{\theta}_{MLE} = \frac{X}{n}\)
\item
  A Bayesian approach with conjugate prior distribution given by
  \(\Theta \sim Beta(a, b)\). The posterior distribution is given by
  \(\Theta | n, x \sim Beta(a + x, b + n - x)\). From this posterior
  distribution, we can obtain point estimates, with the most common
  choice being the posterior mean, which can be written as follows:
\end{enumerate}

\[\hat{\theta}^{Bayes} = \frac{a + x}{(a + x) + (b + n - x)} = \cdots = (1 - w) \frac{a}{a + b} + w\frac{x}{n},\]
where \(w = \frac{n}{n + a + b}\).

By considering what the posterior mean would be across different
samples, we can view the posterior mean as specifying an estimator,

\[\hat{\theta}^{Bayes} = (1 - w) \frac{a}{a + b} + w\frac{X}{n}.\]

In the previous lab, we considered three different prior specifications
for \(\Theta\) with different level of informativeness, and we saw that
for a large sample size it didn't matter which prior we used; the
resulting estimates were essentially the same as each other and as the
maximum likelihood estimate.

Let's now look more closely at how the Bayesian and Frequentist methods
compare for smaller sample sizes. For concreteness, let's work with the
intermediate of our three prior specifications,
\(\Theta \sim \text{Beta}(2, 8)\). Let's also imagine that our sample
size is fixed at 10.

In this case, \(w = \frac{n}{n + a + b} = \frac{10}{10 + 2 + 8} = 0.5\),
and the Bayesian estimator is

\[\hat{\theta}^{Bayes} = (1 - w) \frac{a}{a + b} + w\frac{X}{n} = 0.5 \frac{2}{10} + 0.5\frac{X}{10}\]

\subsection{\texorpdfstring{Properties of
\(\hat{\theta}_{MLE}\)}{Properties of \textbackslash{}hat\{\textbackslash{}theta\}\_\{MLE\}}}\label{properties-of-hattheta_mle}

\paragraph{\texorpdfstring{Find the bias of
\(\hat{\theta}_{MLE}\)}{Find the bias of \textbackslash{}hat\{\textbackslash{}theta\}\_\{MLE\}}}\label{find-the-bias-of-hattheta_mle}

\vspace{3cm}

\paragraph{\texorpdfstring{Find the variance of \(\hat{\theta}_{MLE}\)
in terms of
\(\theta\)}{Find the variance of \textbackslash{}hat\{\textbackslash{}theta\}\_\{MLE\} in terms of \textbackslash{}theta}}\label{find-the-variance-of-hattheta_mle-in-terms-of-theta}

\vspace{3cm}

\paragraph{\texorpdfstring{Find an expression for the MSE of
\(\hat{\theta}_{MLE}\) in terms of
\(\theta\)}{Find an expression for the MSE of \textbackslash{}hat\{\textbackslash{}theta\}\_\{MLE\} in terms of \textbackslash{}theta}}\label{find-an-expression-for-the-mse-of-hattheta_mle-in-terms-of-theta}

\vspace{3cm}

\paragraph{\texorpdfstring{Find the Fisher information about \(\theta\)
from a sample of size
\(n\).}{Find the Fisher information about \textbackslash{}theta from a sample of size n.}}\label{find-the-fisher-information-about-theta-from-a-sample-of-size-n.}

You may use the fact that
\[\frac{d^2}{d \theta^2} \ell(\theta|x) = \frac{-x}{\theta^2} + \frac{-(n - x)}{(1 - \theta)^2}\]

\vspace{3cm}

\paragraph{\texorpdfstring{Is \(\hat{\theta}_{MLE}\) an efficient
estimator of
\(\theta\)?}{Is \textbackslash{}hat\{\textbackslash{}theta\}\_\{MLE\} an efficient estimator of \textbackslash{}theta?}}\label{is-hattheta_mle-an-efficient-estimator-of-theta}

\vspace{3cm}

\paragraph{\texorpdfstring{Suppose \(\tilde{\theta}\) is an unbiased
estimator of \(\theta\). Is it possible that
\(Var(\tilde{\theta}) < Var(\hat{\theta}_{MLE})\)?}{Suppose \textbackslash{}tilde\{\textbackslash{}theta\} is an unbiased estimator of \textbackslash{}theta. Is it possible that Var(\textbackslash{}tilde\{\textbackslash{}theta\}) \textless{} Var(\textbackslash{}hat\{\textbackslash{}theta\}\_\{MLE\})?}}\label{suppose-tildetheta-is-an-unbiased-estimator-of-theta.-is-it-possible-that-vartildetheta-varhattheta_mle}

\newpage

\subsection{\texorpdfstring{Properties of
\(\hat{\theta}^{Bayes}\)}{Properties of \textbackslash{}hat\{\textbackslash{}theta\}\^{}\{Bayes\}}}\label{properties-of-hatthetabayes}

\paragraph{\texorpdfstring{Find the bias of \(\hat{\theta}_{Bayes}\) in
terms of \(\theta\). For what value of \(\theta\) is
\(\hat{\theta}_{Bayes}\) unbiased? How does that relate to the prior
distribution?}{Find the bias of \textbackslash{}hat\{\textbackslash{}theta\}\_\{Bayes\} in terms of \textbackslash{}theta. For what value of \textbackslash{}theta is \textbackslash{}hat\{\textbackslash{}theta\}\_\{Bayes\} unbiased? How does that relate to the prior distribution?}}\label{find-the-bias-of-hattheta_bayes-in-terms-of-theta.-for-what-value-of-theta-is-hattheta_bayes-unbiased-how-does-that-relate-to-the-prior-distribution}

\vspace{3cm}

\paragraph{\texorpdfstring{Find the variance of \(\hat{\theta}_{Bayes}\)
in terms of
\(\theta\)}{Find the variance of \textbackslash{}hat\{\textbackslash{}theta\}\_\{Bayes\} in terms of \textbackslash{}theta}}\label{find-the-variance-of-hattheta_bayes-in-terms-of-theta}

\vspace{3cm}

\paragraph{\texorpdfstring{Find an expression for the MSE of
\(\hat{\theta}_{Bayes}\) in terms of
\(\theta\)}{Find an expression for the MSE of \textbackslash{}hat\{\textbackslash{}theta\}\_\{Bayes\} in terms of \textbackslash{}theta}}\label{find-an-expression-for-the-mse-of-hattheta_bayes-in-terms-of-theta}

\vspace{3cm}

The last part of this lab is on GitHub, Lab 13.


\end{document}
