\documentclass[]{article}
\usepackage{lmodern}
\usepackage{amssymb,amsmath}
\usepackage{ifxetex,ifluatex}
\usepackage{fixltx2e} % provides \textsubscript
\ifnum 0\ifxetex 1\fi\ifluatex 1\fi=0 % if pdftex
  \usepackage[T1]{fontenc}
  \usepackage[utf8]{inputenc}
\else % if luatex or xelatex
  \ifxetex
    \usepackage{mathspec}
  \else
    \usepackage{fontspec}
  \fi
  \defaultfontfeatures{Ligatures=TeX,Scale=MatchLowercase}
\fi
% use upquote if available, for straight quotes in verbatim environments
\IfFileExists{upquote.sty}{\usepackage{upquote}}{}
% use microtype if available
\IfFileExists{microtype.sty}{%
\usepackage{microtype}
\UseMicrotypeSet[protrusion]{basicmath} % disable protrusion for tt fonts
}{}
\usepackage[margin=1in]{geometry}
\usepackage{hyperref}
\hypersetup{unicode=true,
            pdftitle={20180123 - Friend or Foe},
            pdfauthor={Evan L. Ray, adapted from Allan Rossman and Beth Chance},
            pdfborder={0 0 0},
            breaklinks=true}
\urlstyle{same}  % don't use monospace font for urls
\usepackage{graphicx,grffile}
\makeatletter
\def\maxwidth{\ifdim\Gin@nat@width>\linewidth\linewidth\else\Gin@nat@width\fi}
\def\maxheight{\ifdim\Gin@nat@height>\textheight\textheight\else\Gin@nat@height\fi}
\makeatother
% Scale images if necessary, so that they will not overflow the page
% margins by default, and it is still possible to overwrite the defaults
% using explicit options in \includegraphics[width, height, ...]{}
\setkeys{Gin}{width=\maxwidth,height=\maxheight,keepaspectratio}
\IfFileExists{parskip.sty}{%
\usepackage{parskip}
}{% else
\setlength{\parindent}{0pt}
\setlength{\parskip}{6pt plus 2pt minus 1pt}
}
\setlength{\emergencystretch}{3em}  % prevent overfull lines
\providecommand{\tightlist}{%
  \setlength{\itemsep}{0pt}\setlength{\parskip}{0pt}}
\setcounter{secnumdepth}{0}
% Redefines (sub)paragraphs to behave more like sections
\ifx\paragraph\undefined\else
\let\oldparagraph\paragraph
\renewcommand{\paragraph}[1]{\oldparagraph{#1}\mbox{}}
\fi
\ifx\subparagraph\undefined\else
\let\oldsubparagraph\subparagraph
\renewcommand{\subparagraph}[1]{\oldsubparagraph{#1}\mbox{}}
\fi

%%% Use protect on footnotes to avoid problems with footnotes in titles
\let\rmarkdownfootnote\footnote%
\def\footnote{\protect\rmarkdownfootnote}

%%% Change title format to be more compact
\usepackage{titling}

% Create subtitle command for use in maketitle
\newcommand{\subtitle}[1]{
  \posttitle{
    \begin{center}\large#1\end{center}
    }
}

\setlength{\droptitle}{-2em}
  \title{20180123 - Friend or Foe}
  \pretitle{\vspace{\droptitle}\centering\huge}
  \posttitle{\par}
  \author{Evan L. Ray, adapted from Allan Rossman and Beth Chance}
  \preauthor{\centering\large\emph}
  \postauthor{\par}
  \predate{\centering\large\emph}
  \postdate{\par}
  \date{Janary 25, 2018}


\begin{document}
\maketitle

\subsection{Introduction}\label{introduction}

Do children who are less than a year old recognize the difference
between nice, friendly behavior as opposed to mean, unhelpful behavior?
Do they make choices based on such behavior? In a study reported in the
November 2007 issue of Nature, researchers investigated whether infants
take into account an individual's actions towards others in evaluating
that individual as appealing or aversive (Hamlin, Wynn, and Bloom,
2007).

In one component of the study, sixteen 10-month-old infants were shown a
``climber'' character (a piece of wood with ``google'' eyes glued onto
it) that could not make it up a hill in two tries. Then they were
alternately shown two scenarios for the climber's next try, one where
the climber was pushed to the top of the hill by another character
(``friend'') and one where the climber was pushed back down the hill by
another character (``foe''). The infant was alternately shown these two
scenarios several times. Then the child was presented with both pieces
of wood (representing the friend and the foe) and asked to pick one to
play with. Videos of this study are available at websites for the UBC
Center for Infant Cognition Lab
(\url{http://cic.psych.ubc.ca/example-stimuli/}) and the Yale Infant
Cognition Center (\url{https://campuspress.yale.edu/infantlab/}).

The ``climber'' character was a yellow triangle; helper and hinderer
were a red square and a blue circle. Which of the square and the circle
was the helper and which was the hinderer was determined randomly for
each baby. Also randomized were which event (helping or hindering) they
observed first and the positions of helper and hinderer when presented
to the infants (on left or right).

Let the random variable \(X\) denote the number of babies (out of the
total of 16) who preferred the helpful toy.

\paragraph{1. What statistical model would be appropriate for the
distribution of the number of babies who preferred the helpful
toy?}\label{what-statistical-model-would-be-appropriate-for-the-distribution-of-the-number-of-babies-who-preferred-the-helpful-toy}

\vspace{0.5cm}

\(X\) \(\sim \,\)

\paragraph{2. The researchers are interested in the underlying
proportion of 10-month-old infants who express a preference for the
helpful character. How does this relate to the statistical model you
wrote down in part
1?}\label{the-researchers-are-interested-in-the-underlying-proportion-of-10-month-old-infants-who-express-a-preference-for-the-helpful-character.-how-does-this-relate-to-the-statistical-model-you-wrote-down-in-part-1}

\newpage

\paragraph{\texorpdfstring{3. Write down the probability mass function
for \(X\), based on the model you specified in part
1.}{3. Write down the probability mass function for X, based on the model you specified in part 1.}}\label{write-down-the-probability-mass-function-for-x-based-on-the-model-you-specified-in-part-1.}

\vspace{3cm}

\paragraph{\texorpdfstring{4. When they conducted this experiment, the
researchers observed a particular value, \(x\), for the number of babies
in the sample who selected the helpful toy. (I'll tell you what \(x\)
was in a minute, but for now let's stick with using symbols.) Write down
the likelihood function \(\mathcal{L}(p | x)\) for the unknown model
parameter \(p\), conditional on the observed data
\(x\).}{4. When they conducted this experiment, the researchers observed a particular value, x, for the number of babies in the sample who selected the helpful toy. (I'll tell you what x was in a minute, but for now let's stick with using symbols.) Write down the likelihood function \textbackslash{}mathcal\{L\}(p \textbar{} x) for the unknown model parameter p, conditional on the observed data x.}}\label{when-they-conducted-this-experiment-the-researchers-observed-a-particular-value-x-for-the-number-of-babies-in-the-sample-who-selected-the-helpful-toy.-ill-tell-you-what-x-was-in-a-minute-but-for-now-lets-stick-with-using-symbols.-write-down-the-likelihood-function-mathcallp-x-for-the-unknown-model-parameter-p-conditional-on-the-observed-data-x.}

\vspace{4cm}

\paragraph{\texorpdfstring{5. This problem is one example where it will
be easier to work with the log-likelihood instead of the likelihood.
Write down the log-likelihood function
\(L(p | x)\).}{5. This problem is one example where it will be easier to work with the log-likelihood instead of the likelihood. Write down the log-likelihood function L(p \textbar{} x).}}\label{this-problem-is-one-example-where-it-will-be-easier-to-work-with-the-log-likelihood-instead-of-the-likelihood.-write-down-the-log-likelihood-function-lp-x.}

\newpage

\paragraph{\texorpdfstring{6. Prove that the maximum likelihood
estimator of \(p\) is the sample proportion,
\(\hat{p}_{MLE} = \frac{x}{n}\).}{6. Prove that the maximum likelihood estimator of p is the sample proportion, \textbackslash{}hat\{p\}\_\{MLE\} = \textbackslash{}frac\{x\}\{n\}.}}\label{prove-that-the-maximum-likelihood-estimator-of-p-is-the-sample-proportion-hatp_mle-fracxn.}

Your life will be easier if you do this by working with the
log-likelihood function. Your proof has two steps.

\subparagraph{(a) Find a critical
point.}\label{a-find-a-critical-point.}

We know that the maximum of the (log-)likelihood function must occur at
a critical point, where \(\frac{d}{d \, p} L(p | x) = 0\). Calculate the
derivative of the log-likelihood function with respect to \(p\), set the
result equal to \(0\), and solve for the critical point.

\vspace{8cm}

\subparagraph{(b) Verify that the critical point is at a maximum of the
likelihood
function.}\label{b-verify-that-the-critical-point-is-at-a-maximum-of-the-likelihood-function.}

Recall that the critical point you found in part (a) is a local maximum
if the second derivative of the log-likelihood function evaluated at
that critical point is negative, and is guaranteed to be a global
maximum if the second derivative is negative everywhere. Verify that
this condition holds.

\newpage

\paragraph{\texorpdfstring{7. The researchers found that \(x=14\) of the
16 infants in the study selected the nice toy. Find the maximum
likelihood estimate of
\(p\).}{7. The researchers found that x=14 of the 16 infants in the study selected the nice toy. Find the maximum likelihood estimate of p.}}\label{the-researchers-found-that-x14-of-the-16-infants-in-the-study-selected-the-nice-toy.-find-the-maximum-likelihood-estimate-of-p.}


\end{document}
